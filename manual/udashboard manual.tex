% !TEX encoding = IsoLatin9


\documentclass[conference]{IEEEtran}
\usepackage{blindtext, graphicx}
\usepackage[latin9]{inputenc}
\usepackage{hyperref}


% correct bad hyphenation here
\hyphenation{op-tical net-works semi-conduc-tor}


\begin{document}
%
% paper title
% can use linebreaks \\ within to get better formatting as desired
\title{U-Dashboard \\
Manual para el desarrollador}\maketitle




% make the title area
\maketitle
\IEEEpeerreviewmaketitle



\section{\textbf{Introducci�n}}
En el presente documento nos referiremos a temas que nos parecen relevantes para el continuo desarrollo y mantenimiento de U-Dashboard, y que a su vez no se encuentran en el documento de dise�o.
\vspace{15mm}
\section{\textbf{Usuarios y Permisos}}
Usuarios y permisos se encuentran definidos en la base de datos y deben ser administrados desde ahi mismo. Para agregar usuarios, permisos y/o administrarlos seguir los pasos:
\begin{enumerate}
\item Agregar usuario a la tabla user. El id corresponde al RUT, y name al nombre completo de la persona. El RUT no debe tener puntos, pero s� el gui�n.
\item Agregar usuario a la tabla permits. Aqu� user corresponde al RUT de la persona.
\item Para asignar roles a una persona tomar en cuenta lo siguiente:
\begin{itemize}
\item director, visualizer y dcc\_assistant corresponden a flags. Una persona con alguna de estas columnas en 1 corresponde a un director, visualizador o asistente del DCC respectivamente.
\item assistant\_unidad, in\_charge\_unidad, finances\_assistant\_unidad e in\_charge\_unidad\_finances corresponden a texto. Un -1 en alguna de estas columnas indica que el usuario asociado no es asistente, encargado, asistente de finanzas o encargado de finanzas de ninguna unidad. N�meros separados por espacio en alguna de las columnas indican las unidades que se encuentran bajo la jurisdicci�n del usuario. Por ejemplo si la columna assistant\_unidad contiene el texto ''3 5 7'', entonces el usuario es asistente de unidad de las unidades 3, 5 y 7, cuyos nombres pueden verse en la tabla Organization. 	
\end{itemize}
\end{enumerate}

 \vfill
\vspace{5mm}

\section{\textbf{Inicio del sistema}}
Para inicializar el sistema siga las siguientes instrucciones:
\begin{enumerate}
\item Instalar Apache 2.4+ , php 5.6+ y MySQL 5.6+
\item Configurar mod\_rewrite
\begin{itemize}
\item Si es ubuntu''sudo a2enmod rewrite''
\item Luego ''sudo service apache2 restart''
\item editar el archivo ''/etc/apache2/sites-available/ default'' y agregar: \\
$<$Directory ''/var/www/html''$>$\\
AllowOverride All \\
$</$Directory$>$
\end{itemize}

\item Abrir php.ini 
\begin{itemize}
\item  Cambiar linea ''date.timezone=America/Santiago''
\end{itemize}
\item Copiar y pegar la carpeta del proyecto a la ruta donde se alojan los sitios. En este caso ''var/www/html/
\item Editar .htaccess que se encuentra dentro de la carpeta del proyecto. Editar ''RewriteBase /udashboard/'' o el nombre de la carpeta donde se aloje el proyecto.
\item Crear Base de datos en MySQL sin ninguna tabla
\begin{enumerate}
\item Collation: utf8\_unicode\_ci
\end{enumerate}
\item Editar ''/var/www/html/udashboard/application/config/database.php
\begin{itemize}
\item Cambiar los datos respectivos de MySQL: hostname, username, password y database.
\end{itemize}
\item Subir via phpmyadmin o consola archivo limpia.sql que es la estructura de la BD
\begin{itemize}
\item Por consola: mysql -u usuario\_BD -p nombre\_BD $<$ limpia.sql
\item Por phpmyadmin:
\begin{enumerate}
\item seleccionar bd al costado izquierdo.
\item En la barra superior click en import.
\item Seleccionar limpia.sql y clickear go.
\end{enumerate}
\end{itemize}
\item Subir via phpmyadmin o consola archivo first\_data.sql que contiene los m�nimos datos que debe contener la BD para que funcione la aplicaci�n
\begin{itemize}
\item Por consola: mysql -u usuario\_BD -p nombre\_BD $<$ first\_data.sql
\item Por phpmyadmin:
\begin{enumerate}
\item seleccionar bd al costado izquierdo.
\item En la barra superior click en import.
\item Seleccionar first\_data.sql y clickear go.
\end{enumerate}
\end{itemize}

\end{enumerate}
\clearpage

\section{\textbf{Base de datos}}
En el documento de dise�o del proyecto pueden encontrar detalladamente el modelo de la base de datos. Este modelo permite m�s flexibilidad de la que se est� usando, la cual explicaremos a continuaci�n:
\subsection{Tabla Organization }
La tabla \textit{Organization} corresponde a la estructura del departamento, representada como un �rbol. Por ejemplo se tiene la siguiente tabla:
%tabla
\begin{table}[htdp]
\caption{Organization}
\begin{center}
\begin{tabular}{|c|c|c|}
\textbf{id} & \textbf{parent} & \textbf{name} \\ 
 1 & 0 & �rea A \\ 
 2 & 0 & �rea B \\ 
3 & 1 & unidad A \\ 
 4 & 2 & unidad B \\ 
\end{tabular}
\end{center}
\label{Organization}
\end{table}%

En este caso tenemos dos �reas, cada una con una unidad, Unidad A est� bajo �rea A, y unidad B est� bajo �rea B. Este modelo permite mover los nodos del �rbol, cambiado las unidades de �rea, y haciendo que las �reas sean unidades (cuidando que estas no tengan unidades ya que la interfaz no lo permite). Si quisi�ramos que ambas unidades fueran de �rea A simplemente cambiamos en la base de datos el valor de parent para Unidad B (id 4) y cambiamos el valor de parent a 1. Como todas las m�tricas se refieren a su unidad o �rea por su id, cambiar su posici�n en el �rbol organizacional no genera problema.
\subsection{Tabla Metric }
Metric define generalmente algo que se quisiera medir, por ejemplo ''n�mero de personal contratado en el a�o'', en esta tabla se definen las unidades de medida, pero no se asocia la m�trica a ninguna �rea. Esta m�trica podr�a querer medirse para m�s de una �rea, por lo que existe una tabla MetOrg que las asocia. Luego en la tabla Measure se ingresan los valores medidos y se asocian a alg�n id de MetOrg. Ejemplificamos:
\begin{table}[htdp]
\caption{Metric}
\begin{center}
\begin{tabular}{|c|c|c|c|}
\textbf{id} & \textbf{category} & \textbf{unit} & \textbf{name}  \\ 
 1 & productiva & num personas & personal contratado \\ 
\end{tabular}
\end{center}
\label{default}
\end{table}%

\begin{table}[htdp]
\caption{MetOrg}
\begin{center}
\begin{tabular}{|c|c|c|}
\textbf{id} & \textbf{org} & \textbf{metric}  \\ 
 1 & 3 (postgrado) & 1  \\ 
 2 & 4 (pregrado) & 1  \\ 
\end{tabular}
\end{center}
\label{default}
\end{table}%

Una de las ventajas de esto, es que eventualmente se podr�a querer comparar una m�trica, como el personal contratado, para las distintas �reas, y esta estructura lo facilita. 
En nuestra implementaci�n no lo usamos, y la interfaz no permite compartir m�tricas entre �reas. 
\vfill
\section{\textbf{Template y librer�as}}

En el siguiente link podr�n encontrar los template usados para crear las interfaces:\\
\hyperref[http://themeforest.net/item/porto-admin-responsive-html5-template/8539472]{http://themeforest.net/item/porto-admin-responsive-html5-template/8539472}
\vspace{5mm}\\
Para la librer�a de gr�ficos dirigirse a la siguiente p�gina:\\
\hyperref[http://www.flotcharts.org]{http://www.flotcharts.org}




%\appendices
%\section{Proof of the First Zonklar Equation}
%\blindtext
%\begin{thebibliography}{1}

%\bibitem{IEEEhowto:kopka}
%H.~Kopka and P.~W. Daly, \emph{A Guide to \LaTeX}, 3rd~ed.\hskip 1em plus
  %0.5em minus 0.4em\relax Harlow, England: Addison-Wesley, 1999.

%\end{thebibliography}


%\begin{IEEEbiography}[{\includegraphics[width=1in,height=1.25in,clip,keepaspectratio]{picture}}]{John Doe}
%\blindtext
%\end{IEEEbiography}
\end{document}
